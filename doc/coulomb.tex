% TITLE 
%=======================================================
\documentclass[a4paper,titlepage,twoside,fleqn]{article}
%=======================================================
\usepackage[english]{babel}
\usepackage[latin2]{inputenc}
\usepackage{lscape}
\usepackage{rotating}
%---------------------------------------------------
% MATH
\usepackage{amsmath}
\usepackage{amssymb}
\usepackage{dcolumn}
\usepackage{booktabs}
\usepackage{mathrsfs}

%---------------------------------------------------
% GRAPHICX
%\usepackage[dvi-ps]{graphicx}
\usepackage{wrapfig}

%---------------------------------------------------
% BIBLIOGRAPHY
\usepackage{natbib}

%---------------------------------------------------
% PAGE FORMATING
\addtolength{\hoffset}{-1cm}
\addtolength{\textwidth}{2cm}
\usepackage{setspace}
\usepackage{textpos}
\usepackage{color}

%---------------------------------------------------
% SHORTCUTS
\newcolumntype{,}{D{.}{,}{2}}
\newcommand{\citee}[1]{\ensuremath{\scriptsize^{\citenum{#1}}}}
\newcommand{\HRule}{\rule{\linewidth}{0.5mm}}
% Quantum notation
\newcommand{\bra}[1]{\ensuremath{\bigl\langle {#1} \bigl\lvert}}
\newcommand{\ket}[1]{\ensuremath{\bigr\rvert {#1} \bigr\rangle}}
\newcommand{\braket}[2]{\ensuremath{\bigl\langle {#1} \bigl\lvert {#2} \bigr\rangle}}
\newcommand{\tbraket}[3]{\ensuremath{\bigl\langle {#1} \bigl\lvert {#2} \bigl\lvert {#3} \bigr\rangle}}
% Math
\newcommand{\pd}{\ensuremath{\partial}}
\newcommand{\DR}{\ensuremath{{\rm d} {\bf r}}}
\newcommand{\BM}[1]{\ensuremath{\mbox{\boldmath${#1}$}}}
% Chemistry (formulas)
\newcommand{\ch}[2]{\ensuremath{\mathrm{#1}_{#2}}}
% Math 
\newcommand{\VEC}[1]{\ensuremath{\mathrm{\mathbf{#1}}}}
%--------------------------------------------------------------------------------------------------
\begin{document}

\begin{center}
\rule{1cm}{1cm}
\hfill
\rm \Large COULOMB.py $^\copyright$ 2012 
\end{center}
\rule{14cm}{1mm}
% Author and supervisor
\begin{textblock}{5.0}(0,0.2)
\begin{minipage}{1.0\textwidth}
\begin{flushleft} %\large
\textbf{Author:} \newline
Bartosz \textsc{B{\l}asiak} \newline
Wroc{\l}aw University of Technology \newline
\verb+globula@o2.pl+
\end{flushleft}
\end{minipage}
\end{textblock}

\begin{textblock}{3.0}(7,0.2)
\begin{minipage}{1.0\textwidth}
\begin{flushleft} %\large
\textbf{Licence:} \newline
Free useage.
\end{flushleft}
\end{minipage}
\end{textblock}
\hfill\vspace{2 cm}
\rule{14cm}{0.2mm}

\tableofcontents
\clearpage
%\vspace{2 cm}
\section{Introduction}
%
This simple program is designed for calculation of coulomb 
interaction energies between two molecular individuals
in terms of \emph{ab initio} quantum mechanical methods 
at various levels of approximation. It serves the calculations
using the following procedures listed:
%
\begin{enumerate}
   \item Cube Method
   \item Multipole Interaction Energy from 
         Cumulative Atomic Multipole Moments Method (CAMM)
   \item Charge Interaction from Electrostatic Potential 
         Fitting Method (ESP)
   \item First-order electrostatic interaction energy from 
         Hybrid Variational-Perturbtional Interaction Energy
         Decomposition Scheme (EL(1) EDS)
\end{enumerate}
%
The {\sc COULOMB.py} package also provides an opportunity to
calculate coulomb transition energy couplings basing on external 
density matrices loaded from {\rm Gaussian} output files. 

The following tutorial is divided into three main sections (apart from this introduction).
In the first the overall program structure is described. The detailed
description of classes and functions is provided along with the further 
explanation of input file syntax which is very simple. Second section
is devoted to rather very short and dirty introduction or summary\footnote{for 
those who are familiar with the subject} of the basic concepts of evaluation 
of coulomb interaction energy using different aproaches. This section is 
also a listing of what is actually implemented in current version of {\rm COULOMB.py}
and what is not. The last part is an appendix containing the review of auxiliary methods
which are important tools for evaluation of electrostatic interaction energy. 

\section{Installation and program structure}

The program uses modified version of PyQuante suite written in Python as
well as external library for evaluation of two-electron integrals. The
installation processes are the same as in the case of original packages.
Probably to install the modified version of PyQuante you have to simply type
the following command:
\begin{verbatim}
sudo python setup.py install
\end{verbatim}

The program is divided into several files:
\begin{itemize}
\item[\verb+coulomb.py+] contains main function \verb+Main(argv)+ with command 
     line parser
\item[\verb+do.py+] contains the task selection routines in class \verb+DO(PARSER)+ 
\item[\verb+parser.py+] contain a parser \verb+PARSER+ for input files
\item[\verb+run.py+] contains class \verb+RUN+ which is a base class for other classes
     representing methods like \verb+ESP+ and \verb+MULTIP+. \verb+RUN+ represents
     the basic variables that are needed to perform other calculations for one
     molecular system (thus it gathers molecular data, multipole integrals and density 
     matrix). More detailed description of this class is given in the text below.
\item[\verb+esp.py+] contains all procedures for performing ESP fitting
\item[\verb+multip.py+] contains all procedures needed for obtaining multipolar 
     distributions of molecular species (either MMM or CAMM -see below).
\item[\verb+eeleds.py+] contains all the procedures for calculations of electrostatic
     interaction energy in 1-st order from EDS. This class does not inherit from \verb+RUN+.
\item[\verb+util.py+] contains some utility functions. The most important is 
     \verb+read_transition_dmatrix+. Others are timing processing utility \verb+TIMER+,
     and print helpers.
\end{itemize}

\subsection{Main classes}

The most important classes are described below:

\subsubsection{PARSER}
This object contains some defaults which are assumed at the very begining
during the input file reading. The defaults are for e.g. Angstroms as units
for structure input and density matrix file type format (Gaussian) in case
the user would provided an external source of this matrix. It contains also
\verb+self.transition=False+ as well as \verb+state=1+. The method \verb+method+
withdraws the following: method (like HF - in the case then the density matrix
has to be calculated by using PyQuante routines so without providing external
density matrix file), basis set and other keywords: \verb+trans+, \verb+units=[unit]+, 
\verb+mpoints=[no of points per atom for ESP]+, \verb+eint=[method(s)]+, \verb+mtp=[method(s)]+,
\verb+pot=[method]+, \verb+pad=number+ and \verb+print+. Method and basis set are 
provided in the input file similarly as in Gaussian input files (see section \ref{sec:inp}).
Method \verb+molecules+ withdraws molecular coodrinates, multiplicity, charge 
and (optionally) density matrix. It looks for keywords \verb+&MOL+, \verb+DMATRIX=[file]+
and molecular data. The constructor takes only the input file as a path.

\subsubsection{DO(PARSER)}
The obcject of this class has to find out that tasks to perform
and to run appropriate routines. \verb+self.mtp_methods+ 
contains the list of multipole distribution tasks. Available are the two:
Molecular Multipole Moments distribution, MMMs and Cumulative Atomic
Multipole Moments, CAMMs. \verb+self.eint_methods+ contains the lsit
of interaction energy calculation tasks. Available within this kind are
interaction energy from CAMMs, ESP and EEL(1)EDS. Important information
is that the \verb+PARSER+ stores finally a \verb+Molecule+ object from 
PyQuante routines (see class \verb+Molecule+ in file of PyQuante package)
and lists of density external matrices. The constructor takes only the input 
file as a path.

\subsubsection{RUN}
This class contains memorials for molecular and method specific data.
The constructor takes \verb+molecule,basis,method,matrix=0,multInts=0+. 
If multints is set to zero the class \verb+RUN+ calculates density matrix 
(at HF level only) using PyQuante routines. 

\subsubsection{ESP(RUN)}
Constructour takes the arguments:
\begin{verbatim}
def __init__(self, molecule, basis, method, mpot=1000, pot='CAMM', pad=10, 
                   SVD=False, matrix=None, multInts=None, cnt=0.0100, 
                   stat=False, Print=True, transition=False):
    RUN.__init__(self, molecule, basis, method, matrix, multInts)
    ...
\end{verbatim}
The meaning of these arguments is as follows: \verb+mpot+ is a number 
of points \emph{per atom} in which the potential has to be evaluated.
The algorithm uses {\bf random point selection}. \verb+pot+ specifies
the method which potential will be evaluated from. It can be \verb+CAMM+
or \verb+WFN+ (wavefunction). \verb+pad+ is a box padding given in Bohrs,
in which the points will be created. The centre of the box is assumed to 
be the geometrical centre of of the molecule given in \verb+molecule+.
\verb+cnt+ denotes condition number threshold being the limit number for
assessing the real order of distance matrix (the analysis will be performed
when \verb+SVD=True+ only\footnote{{\color{red}not yet written!}}). 
\verb+Print+ statement says if the fitted charges are to be printed out
and \verb+transition+ is to specify if we calculate the transition potential
from transition density matrices (then nuclear contributions to the potential
are zero).

\subsubsection{MULTIP(RUN)}
Constructeur takes the arguments:
\begin{verbatim}
def __init__(self, molecule, basis, method,matrix=None,multInts=None,
                   transition=False):
    RUN.__init__(self, molecule, basis, method,matrix,multInts)
    ...
\end{verbatim} 
The arguments are the same as in \verb+RUN+ class and one additional 
\verb+transition+ has exactly the same meaning as above described 
\verb+ESP+ class. For the purpose of the calculations for this class
several small modifications were made in original PyQuante suite. One of this
is adding multipole integrals formation (\verb+getM+ procedure) and 
adding a memorial \verb+LIST1+ to instances of \verb+bfs+ class which
indicates the number of atom as a function of atomic orbital and is 
an object of type \verb+list+ implemented in Python. For example
\begin{verbatim}
      LIST1 - the list of atoms in the order of basis functions used, 
      e.g.: for h2o molecule with atoms: 8,1,1 and STO-3G basis 
      LIST1 = 0    0    0    0    0    1    2
              |    |    |    |    |    |    |
             1s   2s   2px  2py  2pz  1s   1s
\end{verbatim}
The numbers as atomic indices are maintained in Python's convention
for labeling list, array and string elements from \verb+0+ value.
\verb+LIST1+ memorial is needed when evaluating of CAMMs (see section \ref{sec:aux}). 

\subsubsection{EELEDS}
The constructor takes the arguments as follows:
\begin{verbatim}
   def __init__(self,molecule1,molecule2,basis,method,
                Transition=False,Exchange=False,
                matrixa=None,matrixb=None):
       ...
\end{verbatim}
Statements \verb+molecule1+ and \verb+molecule2+ refer to two molecules, 
for which EEL(1)EDS will be calculated. \verb+Transition+ has the same meaning
as previously. \verb+Exchange+ means exchange term from $\mathscr{K}_{i_1j_1k_2l_2}$ 
integrals. \verb+matrixa+ and \verb+mtrixb+ (optional) are externam density matrices
calculated in DCBS respectively for monomers \verb+molecule1+ and \verb+molecule2+.

\subsection{Input file syntax\label{sec:inp}}

In the input file one can provide more than one molecules to calculate
multipole distributions. If one can obtain interction energy the only
one amount of molecules given is two. The input file philosophy is similar
for Gaussian and Molcas in order to make it more easy to handle. Beneath,
an examplary input file is depicted. 
\begin{figure}
\begin{verbatim}
# CIS/6-31G* EINT=CAMM,ESP  TRANS mpoints=3200

&MOL1
ETHYLENE 1
DMATRIX=e1.cis-6-31Gd.log

0 1
C     -0.031871   -0.000007    0.001312
H     -0.053949   -0.102551    1.069139
C      1.188324    0.000003   -0.695389
H     -0.962724   -0.100044   -0.523396
H      1.209504   -0.099215   -1.763690
H      2.119240   -0.102685   -0.171502

&MOL
ETHYLENE 2 
DMATRIX=e2.cis-6-31Gd.log

0 1
C     -0.031871    3.000007    0.001312
H     -0.053949    3.102551    1.069139
C      1.188324    2.999997   -0.695389
H     -0.962724    3.100044   -0.523396
H      1.209504    3.099215   -1.763690
H      2.119240    3.102685   -0.171502

&END

\end{verbatim}
\caption{Input file for coulomb coupling constant evaluation
         for ethylene dimer obtained at CIS/6-31G* level of theory
         and using two methods: CAMM and ESP. Number of points
         per atom is set to 3200. Density matrices are provided
         for each the monomer. Units are assumed to be in Angstroms 
         (default).}
\end{figure}
There are two parts of input file: methods etc in the form:
\begin{verbatim}
# [method]/[basis set] other commands 
\end{verbatim}
There cannot be any white character in the fragment \verb+[method]/[basis set]+.
Commands are as follows:
\begin{itemize}
\item \verb+MTP=[method1],[method2],...+ where methods are 
      \verb+CAMM+ and \verb+MMM+. Examples are \verb+MTP=MMM+ 
      or \verb+MTP=CAMM,MMM+.
\item \verb+EINT=[method1],[method2],...+ similarly like above. 
      Available methods are \verb+MTP=ESP+, \verb+MTP=CAMM+ and
      \verb+MTP=EELEDS+. Example \verb+MTP=CAMM,ESP,EELEDS+ 
      will compute interaction energy using three methods for 
      the same pair of molecules. {\color{red}IMPORTANT! I didn't
      finished this part so for now the program will be calculatining
      molecular specific variables each time for each method separately
      so it is not so good (e.g. calculates multipole integrals twice 
      for ESP (if potential has to be calculated from CAMM) and CAMM. 
      I will solve this problem soon. I wanted to make class \verb++}
\item \verb+MPOINTS=[integer]+ specifies the number of point per atom
\item \verb+PAD=[real]+ specifies molecular box padding for random 
      points generation
\item \verb+TRANS+ sets transition to be true (requires external 
      transition density matrix - see below)
\item \verb+EXCH+ sets exchange in EELEDS method for interaction
      energy computation
\item \verb+UNITS=[unit]+ specifies units. Available are Angstroms 
      (default) and Bohrs. 
\end{itemize}

The second part of the input file specifies molecule(s)
and has the form:
\begin{verbatim}
&MOL
[molecule name]
DMATRIX=[file with density matrix]

[charge] [multiplicity]
[atom_1]  [x] [y] [z]
[atom_2]  [x] [y] [z]
...
[atom_n]  [x] [y] [z]

/other molecules starting with new &MOL statement!/

&END
\end{verbatim}
The line containing \verb+DMATRIX=+ is of course optional. 
All the input querries are {\rm case insensitive}.  

In order to use {\sc COULOMB.py} one has to simply load the input
file in the following way:
\begin{verbatim}
coulomb.py [input_file]
\end{verbatim}
After this computations start.

\section{Long-range Interaction energy calculations\label{sec:eint}}
\subsection*{Starting from operators}
To derive working formulas for interaction energy calculations
one has to specify first the Hamiltonian $\mathscr{H}_{\rm int}$ operator of interaction
between charge distributions of interest. In principle, these 
interactions are \emph{electrostatic} between all particles in the 
system, i.e. electrons and protons from nuclei. Since the interaction
comes from electrostatics the exact Hamiltonian is pairwise additive
and one can concentrate on the interaction between two molecules, A 
and B. It is important to remark here that this section is devoted to 
treating long-range intermolecular interactions (i.e. without taking 
into account overlap and exchange-repulsion effects explicitly). The 
resulting interaction Hamiltonian can be drawn as follows: 
\begin{equation}
\mathscr{H}_{\rm int} = \mathscr{H}_{\rm int}^{\rm nuc} + \mathscr{H}_{\rm int}^{\rm el-nuc} + \mathscr{H}_{\rm int}^{\rm el} \; ,
\end{equation}
where:
\begin{eqnarray}
\mathscr{H}_{\rm int}^{\rm nuc}    &=&   \sum_{\alpha \in {\rm A}} \sum_{\beta \in {\rm B}} \frac{Z_{\alpha}Z_{\beta}}{r_{\alpha\beta}} \\
\mathscr{H}_{\rm int}^{\rm el-nuc} &=& - \sum_{\alpha \in {\rm A}} \sum_{j \in {\rm B}} \frac{Z_{\alpha}}{r_{\alpha j}}
                                       - \sum_{\beta  \in {\rm B}} \sum_{i \in {\rm A}} \frac{Z_{\beta }}{r_{\beta  i}} \\
\mathscr{H}_{\rm int}^{\rm el}     &=& \sum_{i  \in {\rm A}} \sum_{j \in {\rm B}} \frac{1}{r_{ij}} 
\end{eqnarray}
In the above notation greek subscripts denote nuclei 
whereas latin subscripts distinguish electrons.
An alternative but very useful form of $\mathscr{H}_{\rm int}$ can be 
achieved using a charge density operator of molecule A, $\hat{\rho}^{\rm A}$
which is defined as:
\begin{equation}\label{e:chd}
\hat{\rho}^{\rm A}(\VEC{r}) = \sum_{a \in {\rm A}} e_a \delta(\VEC{r}-\VEC{a}) \; , 
\end{equation}
where the summation extends over all the particles (nuclei and electrons).
Using \ref{e:chd} and integrating out nuclear charge densities\footnote{we use Born-Oppenheimer approximation} 
the respective interaction Hamiltonian terms can be written 
as follows:
\begin{eqnarray}\label{e:rho}
\mathscr{H}_{\rm int}^{\rm nuc}    &=&   \sum_{\alpha \in {\rm A}} \sum_{\beta \in {\rm B}} \frac{Z_{\alpha}Z_{\beta}}{r_{\alpha\beta}} \\
\mathscr{H}_{\rm int}^{\rm el-nuc} &=& - \sum_{\alpha \in {\rm A}} Z_{\alpha} \int \frac{\hat{\rho}^{\rm B}(\VEC{r}_j)}{r_{\alpha j}} {\rm d}\VEC{r}_j
                                       - \sum_{\beta  \in {\rm B}} Z_{\beta}  \int \frac{\hat{\rho}^{\rm A}(\VEC{r}_i)}{r_{\beta  i}} {\rm d}\VEC{r}_i \\
\mathscr{H}_{\rm int}^{\rm el}     &=& \iint \frac{\hat{\rho}^{\rm A}(\VEC{r}_i)\hat{\rho}^{\rm B}(\VEC{r}_j)}{r_{ij}} 
                                       {\rm d}\VEC{r}_j {\rm d}\VEC{r}_i \quad (i\in A \;{\rm and}\;  j\in B)\; ,
\end{eqnarray} 
where $\hat{\rho}^{\rm A}(\VEC{r}_i)$ are in this case only electron 
charge density operators.

There is another approach, also sometimes very useful, based on 
multipole expansion. The interaction Hamiltonian can be written
in the form:
\begin{equation}
\mathscr{H}_{\rm int} = \sum_{a\in {\rm A}} e_a V_{\rm B}(\mathbf a)  = \sum_{b\in {\rm B}} e_b V_{\rm A}(\mathbf b)\; ,
\end{equation}
where $V_{\rm X}(\mathbf r)$ is a electrostatic potential in point $\mathbf r$ 
generated by molecule X. Expanding this potential in Taylor series 
and applying traceless multipole moments the interaction Hamiltonian
has the form (up to hexadecapole moments):
\begin{equation}\label{e:hmult}
\mathscr{H}_{\rm int} = \hat{q}^B\hat{V}^A + \hat{\mu}^B_a\hat{V}^A_a + \frac{1}{3} \hat{\Theta}^B_{ab} \hat{V}^A_{ab} 
                + \frac{1}{15} \hat{\Omega}^B_{abc} \hat{V}^A_{abc} 
                + \frac{1}{105} \hat{\Xi}^B_{abcd} \hat{V}^A_{abcd} \; ,
\end{equation}
where potential is expressed by a series of $T$-tensors (interaction tensors):
\begin{equation}\label{e:potmult}
\hat{V}^X = \hat{q}^X \hat{T} + \hat{\mu}^X_a\hat{T}_a + \frac{1}{3} \hat{\Theta}^X_{ab}\hat{T}_{ab} 
                              + \frac{1}{15} \hat{\Omega}^X_{abc} \hat{T}_{abc} 
                              + \frac{1}{105} \hat{\Xi}^B_{abcd} \hat{T}_{abcd}
\end{equation}
and
\begin{eqnarray}
\hat{T}              &=& \frac{1}{R} \\
\hat{T}_{ab\cdots x} &=& \nabla_a\nabla_b\cdots\nabla_x \Big(\frac{1}{R}\Big) 
\end{eqnarray}
The potential gradient operators and higher-order derivatives $\hat{V}_{ab\cdots x}$
are related very simply with the potential operator $\hat{V}$:
\begin{equation}
\hat{V}_{ab\cdots x} = \nabla_a\nabla_b\cdots\nabla_x \hat{V} 
\end{equation}
Substituting into \ref{e:hmult} the terms of potential and its derivatives
and keeping in mind that we expand the interaction between A and B
with respect to ${\mathbf R} = {\mathbf R}_{\rm B} - {\mathbf R}_{\rm A}$
(so we have to change the sign of the odd-rank $\hat{T}$-tensor operators for the potential in \ref{e:potmult}) 
we obtain\footnote{we take into account terms up to 4-th rank}:
\begin{eqnarray}
\mathscr{H}_{\rm int}^{\rm} 
&=& \hat{q}^B \big[ 
                 \hat{q}^A\hat{T} - 
                 \hat{\mu}^A_a\hat{T}_a + 
                 \frac{1}{3} \hat{\Theta}^A_{ab}\hat{T}_{ab} -
                 \frac{1}{15} \hat{\Omega}^A_{abc} \hat{T}_{abc} + 
                 \frac{1}{105} \hat{\Xi}^A_{abcd} \hat{T}_{abcd} 
              \big] \\ \nonumber
&+& \hat{\mu}^B_a \big[  
                      \hat{q}^A\hat{T}_a - 
                      \hat{\mu}^A_b\hat{T}_{ab} +  
                      \frac{1}{3} \hat{\Theta}^A_{bc}\hat{T}_{abc} -  
                      \frac{1}{15} \hat{\Omega}^A_{bcd} \hat{T}_{abcd} 
                  \big] \\ \nonumber
&+& \frac{1}{3}\hat{\Theta}^B_{ab} \big[ 
                                          \hat{q}^A\hat{T}_{ab} - 
                                          \hat{\mu}^A_c\hat{T}_{abc} + 
                                          \frac{1}{3} \hat{\Theta}^A_{cd}\hat{T}_{abcd} 
                                   \big] \\ \nonumber
&+& \frac{1}{15}\hat{\Omega}^B_{abc} \big[ 
                                            \hat{q}^A\hat{T}_{abc} - 
                                            \hat{\mu}^A_d\hat{T}_{abcd} 
                                     \big] \\ \nonumber
&+& \frac{1}{105}\hat{\Xi}^B_{abcd} \big[ 
                                           \hat{q}^A\hat{T}_{abcd}  
                                    \big] 
\end{eqnarray}
The above experesion can be rearranged into a more convenient form 
containing respective $\hat{T}$-tensor terms:
\begin{eqnarray}\label{e:mtp}
\mathscr{H}_{\rm int}^{\rm} 
&=& \hat{T} \hat{q}^A \hat{q}^B \\ \nonumber
&+& \hat{T}_a \big(
                     \hat{\mu}^B_a \hat{q}^A -
                     \hat{q}^B \hat{\mu}^A_a 
              \big) \\ \nonumber
&+& \hat{T}_{ab} \big( 
                       \frac{1}{3} \hat{q}^B \hat{\Theta}^A_{ab} + 
                       \frac{1}{3} \hat{q}^A \hat{\Theta}^B_{ab} - 
                       \hat{\mu}_b^A \hat{\mu}_a^B 
                 \big)\\ \nonumber
&+& \hat{T}_{abc} \big( \frac{1}{3} \hat{\mu}_a^B \hat{\Theta}^A_{bc} - 
                        \frac{1}{3} \hat{\Theta}^B_{ab} \hat{\mu}_c^A -
                        \frac{1}{15} \hat{\Omega}^A_{abc} \hat{q}^B +
                        \frac{1}{15} \hat{\Omega}^B_{abc} \hat{q}^A 
                   \big)\\ \nonumber
&+& \hat{T}_{abcd} \big( 
                        \frac{1}{9} \hat{\Theta}^B_{ab} \hat{\Theta}^A_{cd} - 
                        \frac{1}{15} \hat{\mu}_a^B \hat{\Omega}^A_{bcd} -
                        \frac{1}{15} \hat{\Omega}^B_{abc} \hat{\mu}_d^A +
                        \frac{1}{105} \hat{\Xi}^A_{abcd} \hat{q}^B + 
                        \frac{1}{105} \hat{q}^A \hat{\Xi}^B_{abcd} 
                   \big)\\ \nonumber
&=& \mathscr{A}_0 + \mathscr{A}_1 + \mathscr{A}_2 + \mathscr{A}_3 + \mathscr{A}_4
\end{eqnarray}
This Hamiltonian formulation will be used to obtain 
the working expression by rewriting interaction tensors 
explicitly in the subsection \ref{s:cammint}.


%\subsection*{Variational treatment of intermolecular interactions}
%The use of variational method is more seldom but it is also used.
%To calculate interaction energy one can treat the orbitals from
%interacting molecules variationally by optimization them in the
%same vector space. Variational calculations can be performed at
%e.g. Hartree-Fock level in order to obtain density matrix which
%in turn is used to calculate resulting interaction energy.  
%The main problem here is the fact that in order to variationally
%optimize interaction energy one have to perform calculation in
%joint Hilbert space of molecule A and B which increases the computational
%cost.

\subsection*{Perturbative treatment of intermolecular interactions}
To calculate interaction energy one uses perturbation theory 
due to the fact that intermolecular interaction can be
regarded as a perturbation of a total Hamiltonian for 
the two molecular systems, $\mathscr{H}=\mathscr{H}_{\rm A}+\mathscr{H}_{\rm B}+\mathscr{H}_{\rm int}$, 
where the first two terms accounts for unperturbed Hamiltonians
of molecules A and B (as if they did not interact) and
the last term defined as in previous paragraph describing
the A--B interaction. 

From the Raileygh-Schr{\"o}dinger perturbation theory one can
calculate the first-order interaction energy which equals:
\begin{equation}
E^{(1)}_{\rm int} = \tbraket{\Psi_{\rm A}\Psi_{\rm B}}
                            {\mathscr{H}_{\rm int}}
                            {\Psi_{\rm A}\Psi_{\rm B}}
\end{equation}
To obtain interaction energy in the first order it has only
to replace the operators from equations 23--28 by
their respective expectation values between ground state 
functions 
%(because $\tbraket{\Psi_{\rm A}\Psi_{\rm B}}
%                            {\hat{O}}
%                            {\Psi_{\rm A}\Psi_{\rm B}}=
%              \hat{O}\braket{\Psi_{\rm A}\Psi_{\rm B}}
%                            {\Psi_{\rm A}\Psi_{\rm B}}=O$
%for any tensor operator $\hat{O}$ in our case).
In order to calculate higher order corrections the expectation
values of these tensor operators between functions including 
also excited states have to be considered. In {\sc COULOMB.py}
only first-order interaction energies are available at present.

Using derived interaction Hamiltonian operator forms and applying
perturbation theory to the first order one can obtain working
formulae for the interaction energy (in the first-order) from 
Density Cube Method (IE DC), first-order electrostatics from 
Variational-perturbational Interaction Energy Decomposition 
Scheme (VP IEDS), Cumulative Atomic Multipole Moments Interaction 
Energy Decomposition Scheme (CAMM IEDS) and Interaction Energy 
from Electrostatic Potential Charge Fitting (IE ESP). Beneath, 
the respective derivations for above mentioned methods are listed.

\subsection{Density Cube Method}

Changing the expectation values of the electron density operators
from \ref{e:rho} one can obtain the following expression for the
first-order interaction energy:
\begin{equation}\label{e:DCM}
E_{\rm int}^{(1)}= \sum_{\alpha \in {\rm A}} \sum_{\beta \in {\rm B}} \frac{Z_{\alpha}Z_{\beta}}{r_{\alpha\beta}} \\
                 - \sum_{\alpha \in {\rm A}} Z_{\alpha} \int \frac{\rho^{\rm B}(\VEC{r}_j)}{r_{\alpha j}} {\rm d}\VEC{r}_j
                 - \sum_{\beta  \in {\rm B}} Z_{\beta}  \int \frac{\rho^{\rm A}(\VEC{r}_i)}{r_{\beta  i}} {\rm d}\VEC{r}_i \\
                 + \iint \frac{\rho^{\rm A}(\VEC{r}_i)\rho^{\rm B}(\VEC{r}_j)}{r_{ij}} {\rm d}\VEC{r}_j {\rm d}\VEC{r}_i
\end{equation} 
This is the concept of Density Cube Method. 
%Resulting integrals 
%should be evaluated numerically from \verb+cube+ files of electron 
%density using trilinear of tricubic quadratures. In {\sc COULOMB.py}
%only trilinear quadrature is impelemented. 
In practice one uses \verb+cube+ files of electron density.

It has been shown that the method doesn't work at all when 
the density cubes overlap. 

\subsection{First-order electrostatics from VP IEDS}

The same equations \ref{e:rho} can be rewritten in terms of
electron density matrix obtained e.g. HF calculations and
the optimized orbitals. Electron density in Restricted Hartee-Fock
(RHF) scheme can be expressed by:
\begin{equation}\label{e:dmat}
\rho(\mathbf{r}) = \sum_{\mu\nu} P_{\mu\nu} \phi_{\mu}^{*}(\mathbf{r})  \phi_{\nu}(\mathbf{r})
\end{equation}   
Substituting \ref{e:dmat} into the integrals from \ref{e:DCM} 
which consequently equal:
\begin{eqnarray}
-Z_{\iota}\int \frac{\rho^X({\bf r}_i)  }{r_{\iota i}} \;{\rm d}{\bf r}_i 
&=&
-Z_{\iota}\int \Big[
     \sum_{\mu\nu} P^X_{\mu\nu} \phi_{\mu}^{*} \frac{1}{r_{\iota i}} \phi_{\nu} 
     \Big] \;{\rm d}{\bf r}_i = \\ \nonumber
&=&
\sum_{\mu\nu} P^X_{\mu\nu} 
 \Big[
 \int -Z_{\iota}\phi_{\mu}^{*} \frac{1}{r_{\iota i}} \phi_{\nu}\;{\rm d}{\bf r}_i
 \Big] %= \\ \nonumber
=
\sum_{\mu\nu} P^X_{\mu\nu} V_{\mu\nu}^{\iota} \\\nonumber 
&&\textrm{  and}\\
%
\iint \frac{\rho^X({\bf r}_i)\rho^Y({\bf r}_j)  }{r_{ij}} \;{\rm d}{\bf r}_i \;{\rm d}{\bf r}_j 
&=&
\iint \Big[
       \sum_{\mu\nu} P^X_{\mu\nu} \phi_{\mu}^{*} \phi_{\nu} 
       \frac{1}{r_{ij}} 
       \sum_{\lambda\sigma} P^Y_{\lambda\sigma} \phi_{\lambda}^{*} \phi_{\sigma}
       \Big] 
       \;{\rm d}{\bf r}_i \;{\rm d}{\bf r}_j = \\ \nonumber
&=&
\sum_{\mu\nu\lambda\sigma} P^X_{\mu\nu} P^Y_{\lambda\sigma} 
   \Big[
     \iint \phi_{\mu}^{*} \phi_{\nu} 
           \frac{1}{r_{ij}} 
           \phi_{\lambda}^{*} \phi_{\sigma}
           \;{\rm d}{\bf r}_i \;{\rm d}{\bf r}_j
   \Big] 
=
\sum_{\mu\nu\lambda\sigma} P^X_{\mu\nu} P^Y_{\lambda\sigma}
(\mu\nu|\lambda\sigma)
\end{eqnarray}
we arrive into the final $E_{\rm el}^{(1)}$ expression:
\begin{equation}\label{e:eeleds}
E_{\rm el}^{(1)} = 
\sum_{\alpha \in {\rm A}} \sum_{\beta \in {\rm B}} \frac{Z_{\alpha}Z_{\beta}}{r_{\alpha\beta}} +
\sum_{\mu\nu\beta} P^A_{\mu\nu} V_{\mu\nu}^{\beta} +
\sum_{\lambda\sigma\alpha} P^B_{\lambda\sigma} V_{\lambda\sigma}^{\alpha} +
\sum_{\mu\nu\lambda\sigma} P^A_{\mu\nu} P^B_{\lambda\sigma} (\mu\nu|\lambda\sigma)
\end{equation} 
The resulting equation use RHF density matrices for 
molecules A and B calculated in Dimer Centered Basis Set 
(DCBS) in order to prevent non-physical effect of Basis 
Set Superposition Error (BSSE). There are also used 
nuclear attraction integrals $V_{\mu\nu}^{\beta}$ 
of electron from one molecule and nuclei from other as
well as two-electron integrals $(\mu\nu|\lambda\sigma)$.
All the integrals are used in Atomic Orbital (AO) 
representation.  


\subsection{Multipole part of electrostatic energy from CAMM IEDS\label{s:cammint}}

Substituting multpipole operators by their respective 
expectation values in \ref{e:mtp} and summing over all
distributed multipole moments in molecules A and B one 
obtains:
\begin{eqnarray}
E_{\rm int}^{(1)}
&=& \sum_{\alpha \in {\rm A}} \sum_{\beta \in {\rm B}} \Big[  
           {T} {q}^{\alpha} {q}^{\beta} \\ \nonumber
&+& {T}_a \big(
                     {\mu}^{\beta}_a {q}^{\alpha} -
                     {q}^{\beta} {\mu}^{\alpha}_a 
              \big) \\ \nonumber
&+& {T}_{ab} \big( 
                       \frac{1}{3} {q}^{\beta} {\Theta}^{\alpha}_{ab} + 
                       \frac{1}{3} {q}^{\alpha} {\Theta}^{\beta}_{ab} - 
                       {\mu}_b^{\alpha} {\mu}_a^{\beta} 
                 \big)\\ \nonumber
&+& {T}_{abc} \big( \frac{1}{3} {\mu}_a^{\beta} {\Theta}^{\alpha}_{bc} - 
                        \frac{1}{3} {\Theta}^{\beta}_{ab} {\mu}_c^{\alpha} -
                        \frac{1}{15} {\Omega}^{\alpha}_{abc} {q}^{\beta} +
                        \frac{1}{15} {\Omega}^{\beta}_{abc} {q}^{\alpha} 
                   \big)\\ \nonumber
&+& {T}_{abcd} \big( 
                        \frac{1}{9} {\Theta}^{\beta}_{ab} {\Theta}^{\alpha}_{cd} - 
                        \frac{1}{15} {\mu}_a^{\beta} {\Omega}^{\alpha}_{bcd} -
                        \frac{1}{15} {\Omega}^{\beta}_{abc} {\mu}_d^{\alpha} +
                        \frac{1}{105} {\Xi}^{\alpha}_{abcd} {q}^{\beta} + 
                        \frac{1}{105} {q}^{\alpha} {\Xi}^{\beta}_{abcd} 
                   \big)  
 \Big] \\ \nonumber
&=& \sum_{\alpha \in {\rm A}} \sum_{\beta \in {\rm B}} \Big[ 
                                                       {A}^{\alpha\beta}_0 + {A}^{\alpha\beta}_1 + 
                                                       {A}^{\alpha\beta}_2 + {A}^{\alpha\beta}_3 + {A}^{\alpha\beta}_4
                                                       \Big]
\end{eqnarray}
The terms from $A^{\alpha\beta}_0$ to $A^{\alpha\beta}_4$ 
have to be calculated further by using interaction tensors 
in their explicit forms:
\begin{eqnarray}\label{e:T}
T       &=& \frac{1}{R} \\
T_a     &=& \nabla_a\frac{1}{R} = -\frac{R_a}{R^3} \\
T_{ab}  &=& \nabla_a\nabla_b\frac{1}{R} = \frac{3R_aRb-R^2\delta_{ab}}{R^5} \\
T_{abc} &=& \nabla_a\nabla_b\nabla_c\frac{1}{R} = 
            \frac{15R_aR_bR_c-3R^2(R_a\delta_{bc}+R_b\delta_{ac}+R_c\delta_{ab})}{R^7} \\
T_{abcd}&=& \nabla_a\nabla_b\nabla_c\nabla_d\frac{1}{R} =\\ \nonumber
        &=& \frac{1}{R^9} 
            \Big[ 105R_aR_bR_cR_d \\ \nonumber
        & &     - 15R^2(R_aR_b\delta_{cd} + R_aR_c\delta_{bd} + R_aR_d\delta_{bc} +
                  R_bR_c\delta_{ad} + R_bR_d\delta_{ac} + R_cR_d\delta_{ab}) \\ \nonumber
        & &     + 3R^4(\delta_{ab}\delta_{cd} + \delta_{ac}\delta_{bd} + \delta_{ad}\delta_{bc}) 
            \Big]
\end{eqnarray}

\subsubsection*{$A^{\alpha\beta}_0$ term}

This term describes the interaction between two monopoles
(charges) and it's very easy:
\begin{equation}
A^{\alpha\beta}_0= \frac{q^{\alpha}q^{\beta}}{R} \;, 
\end{equation}  
where $R=\big|{\bf R}_{\beta}-{\bf R}_{\alpha}\big|$.

\subsubsection*{$A^{\alpha\beta}_1$ term}

This term describes the interaction between monopole and
dipole and equals:
\begin{equation}
A^{\alpha\beta}_1= -\frac{R_a}{R^3} 
\Big[ 
    \mu^B_a q^A -
    q^B \mu^A_a 
\Big] = \frac{q^{\beta}{\BM \mu}^{\alpha}{\bf R} - q^{\alpha}{\BM \mu}^{\beta}{\bf R}}   {R^3} 
\end{equation}  

\subsubsection*{$A^{\alpha\beta}_2$ term}

This term describes the interactions between monopole 
and quadrupole as well as dipole-dipole interactions:
\begin{eqnarray}
A^{\alpha\beta}_2&=& \frac{3R_aR_b-R^2\delta_{ab}}{R^5} 
\Big[ 
    \frac{1}{3} {q}^{\beta} {\Theta}^{\alpha}_{ab} + 
    \frac{1}{3} {q}^{\alpha} {\Theta}^{\beta}_{ab} - 
    {\mu}_b^{\alpha} {\mu}_a^{\beta}  
\Big] = \frac{{\BM \mu}^{\alpha}{\BM \mu}^{\beta}}{R^3} - \\ \nonumber
&-& 
3\frac{({\bf R}{\BM \mu}^{\alpha})({\bf R}{\BM \mu}^{\beta})}{R^5} +
\frac{q^{\beta}({\bf R}\otimes_1{\bf\Theta}^{\alpha}{}_2\otimes{\bf R}) + q^{\alpha}({\bf R}\otimes_1{\bf\Theta}^{\beta}{}_2\otimes{\bf R})  }{R^5} 
\end{eqnarray}  

\subsubsection*{$A^{\alpha\beta}_3$ term}

This term describes the interactions between dipole 
and quadrupole as well as monopole-octupole interactions
and equals:
\begin{eqnarray}
A^{\alpha\beta}_3&=& \frac{15R_aR_bR_c-3R^2(R_a\delta_{bc}+R_b\delta_{ac}+R_c\delta_{ab})}{R^7} \times \\ \nonumber
& & \quad\times \Big[
           \frac{1}{3} {\mu}_a^{\beta} {\Theta}^{\alpha}_{bc} - 
           \frac{1}{3} {\Theta}^{\beta}_{ab} {\mu}_c^{\alpha} -
           \frac{1}{15} {\Omega}^{\alpha}_{abc} {q}^{\beta} +
           \frac{1}{15} {\Omega}^{\beta}_{abc} {q}^{\alpha}
           \Big] = \\ \nonumber
&=&
5\frac{R_c\mu_c^{\alpha} \cdot R_a\Theta_{ab}^{\beta}R_b - R_a\mu_a^{\beta} \cdot R_b\Theta_{bc}^{\alpha}R_c}{R^7} +
 \frac{ q^{\beta}R_aR_bR_c\Omega^{\alpha}_{abc} - q^{\alpha}R_aR_bR_c\Omega^{\beta}_{abc}    }{R^7} + \\ \nonumber
&&\quad+ 
\frac{\mu_a^{\beta}R_a\Theta_{bb}^{\alpha} + R_b\Theta_{bc}^{\alpha}\mu_c^{\beta} + \mu_b^{\beta}\Theta_{bc}^{\alpha}R_c }{R^5} -
\frac{R_a\Theta_{ac}^{\beta}\mu_c^{\alpha} + \mu_c^{\alpha}\Theta_{cb}^{\beta}R_b + \mu_c^{\alpha}R_c\Theta_{bb}^{\beta} }{R^5} - \\ \nonumber
&&\quad- 
\frac{q^{\beta}(R_a\Omega_{acc}^{\alpha} + R_b\Omega_{cbc}^{\alpha} + R_c\Omega_{bbc}^{\alpha})  }{R^5} +
\frac{q^{\alpha}(R_a\Omega_{acc}^{\beta} + R_b\Omega_{cbc}^{\beta} + R_c\Omega_{bbc}^{\beta})  }{R^5} = \\ \nonumber
&=&
5\frac{{\bf R}{\BM\mu}^{\alpha}( {\bf R}\otimes_1{\BM\Theta}^{\beta}{}_2\otimes{\bf R}) - 
       {\bf R}{\BM\mu}^{\beta} ( {\bf R}\otimes_1{\BM\Theta}^{\alpha}{}_2\otimes{\bf R})  }{R^7} + \\ \nonumber
&&\quad+
\frac{ q^{\beta}({\bf R}\otimes_1{\bf R}\otimes_2{\BM\Omega}^{\alpha}{}_3\otimes{\bf R}) - 
       q^{\alpha}({\bf R}\otimes_1{\bf R}\otimes_2{\BM\Omega}^{\beta}{}_3\otimes{\bf R})    }{R^7} + \\ \nonumber
&&\quad+
2\frac{   {\BM \mu}^{\beta}\otimes_1{\BM\Theta}^{\alpha}{}_2\otimes{\bf R} - 
          {\BM \mu}^{\alpha}\otimes_1{\BM\Theta}^{\beta}{}_2\otimes{\bf R}   }{R^5}
\end{eqnarray}
In the following derivations we used the traceless 
property of quadrupole and octupole moments, i.e.
$\Theta_{aa}=0$ and $\Omega_{aab}=\Omega_{aba}=\Omega_{baa}=0$
for all combinations of subscripts.

\subsubsection*{$A^{\alpha\beta}_4$ term}

This term describes the interactions between dipole 
and octupole, quadrupole-quadrupole as well as 
monopole and hexadecapole interactions:
\begin{eqnarray}
A^{\alpha\beta}_4&=& \frac{1}{R^9} 
            \Big[ 105R_aR_bR_cR_d \\ \nonumber
        & &     - 15R^2(R_aR_b\delta_{cd} + R_aR_c\delta_{bd} + R_aR_d\delta_{bc} +
                  R_bR_c\delta_{ad} + R_bR_d\delta_{ac} + R_cR_d\delta_{ab}) \\ \nonumber
        & &     + 3R^4(\delta_{ab}\delta_{cd} + \delta_{ac}\delta_{bd} + \delta_{ad}\delta_{bc}) 
            \Big] \times \\ \nonumber
&&\quad\times\Big[ 
\frac{1}{9} {\Theta}^{\beta}_{ab} {\Theta}^{\alpha}_{cd} - 
\frac{1}{15} {\mu}_a^{\beta} {\Omega}^{\alpha}_{bcd} -
\frac{1}{15} {\Omega}^{\beta}_{abc} {\mu}_d^{\alpha} +
\frac{1}{105} {\Xi}^{\alpha}_{abcd} {q}^{\beta} + 
\frac{1}{105} {q}^{\alpha} {\Xi}^{\beta}_{abcd}  
\Big] =  \\ \nonumber 
&=&
\frac{35}{3} \frac{R_a\Theta_{ab}^{\alpha}R_b\cdot R_c\Theta_{cd}^{\beta}R_d }{R^9}     -
7\frac{\mu_a^{\beta}R_a \cdot R_bR_cR_d\Omega_{bcd}^{\alpha} + 
       R_aR_bR_c\Omega_{abc}^{\beta} \cdot \mu_d^{\alpha}R_d  }{R^9} - \\ \nonumber
&&\quad- 
\frac{15}{9}
\frac{R_a\Theta_{ad}^{\beta}\Theta_{dc}^{\alpha}R_c + R_b\Theta_{bd}^{\beta}\Theta_{dc}^{\alpha}R_c +
      R_b\Theta_{ba}^{\beta}\Theta_{ad}^{\alpha}R_d + R_a\Theta_{ac}^{\beta}\Theta_{cd}^{\alpha}R_d   }{R^7} + \\ \nonumber
&&\quad+
\frac{R_bR_c\Omega_{bca}^{\alpha}\mu_a^{\beta} + 
      R_cR_d\Omega_{cda}^{\alpha}\mu_a^{\beta} +
      R_bR_d\Omega_{bda}^{\alpha}\mu_a^{\beta}}{R^7}  + \\ \nonumber
&&\quad+
\frac{R_bR_c\Omega_{bca}^{\beta}\mu_a^{\alpha} + 
      R_cR_d\Omega_{cda}^{\beta}\mu_a^{\alpha} +
      R_bR_d\Omega_{bda}^{\beta}\mu_a^{\alpha}}{R^7}  + \\ \nonumber
&&\quad+
\frac{1}{3}
\frac{\Theta_{cd}^{\beta}\Theta_{cd}^{\alpha} + \Theta_{cd}^{\beta}\Theta_{cd}^{\alpha}}{R^5} +
\frac{R_aR_bR_cR_d(q^{\alpha}\Xi_{abcd}^{\beta}+\Xi_{abcd}^{\alpha}q^{\beta} )}{R^9} + \\ \nonumber
&&\quad+
\;\textrm{remaining zero terms with traces of }\Omega\;\textrm{and }\Xi= \\ \nonumber
&=&
\frac{35}{3} \frac{
({\bf R}\otimes_1{\BM\Theta}^{\alpha}{}_2\otimes{\bf R})\cdot ({\bf R}\otimes_1{\BM\Theta}^{\beta}{}_2\otimes{\bf R})
 }{R^9} - \\ \nonumber
&&\quad-
7\frac{
({\BM\mu^{\beta}}{\bf R}) \cdot ({\bf R}\otimes_1{\bf R}\otimes_2{\BM\Omega}^{\alpha}{}_3\otimes{\bf R}) +
({\BM\mu^{\alpha}}{\bf R}) \cdot ({\bf R}\otimes_1{\bf R}\otimes_2{\BM\Omega}^{\beta}{}_3\otimes{\bf R})
}{R^9} - \\ \nonumber
&&\quad-
\frac{20}{3}
\frac{
({\bf R}\otimes_1{\BM\Theta}^{\alpha})\cdot({\bf R}\otimes_1{\BM\Theta}^{\beta})
}{R^7} +
\frac{2}{3}
\frac{({\BM\Theta}^{\alpha}{}_{12}\otimes_{12}{\BM\Theta}^{\beta})}{R^5} + \\ \nonumber
&&\quad+
3\frac{
({\bf R}\otimes_1{\bf R}\otimes_2{\BM\Omega}^{\alpha}{}_3\otimes{\BM \mu^{\beta}})+
({\bf R}\otimes_1{\bf R}\otimes_2{\BM\Omega}^{\beta}{}_3\otimes{\BM \mu^{\alpha}})
}{R^7} + \\ \nonumber
&&\quad+
\frac{
q^{\beta} ({\bf R}\otimes_1{\bf R}\otimes_2{\bf R}\otimes_3{\BM\Xi}^{\alpha}{}_4\otimes{\bf R}) +
q^{\alpha} ({\bf R}\otimes_1{\bf R}\otimes_2{\bf R}\otimes_3{\BM\Xi}^{\beta}{}_4\otimes{\bf R})  
}{R^9}
\end{eqnarray}
%
These working formulae are implemented in {\sc COULOMB.py} 
standard routines except for the terms containing 
hexadecapole moments. 

\subsection{Electrostatic energy from ESP}

Having the charges fitted by applying ESP procedure the first-order
interaction energy simply equals:
\begin{equation}
E_{\rm int}^{(1)} = \sum_{\alpha\in A}\sum_{\beta\in B}\frac{q^{\alpha}q^{\beta}}{r_{\alpha\beta}}
\end{equation}  

\subsection{DFI pseudo-coulomb interaction energy}

DFI method differs strongly from the previous ones primarily 
because it involves not only classical electrostatic interaction 
but also intermolecular first-order exchange-repulsion (from
exchange integrals between electrons of respective molecules 
or, here, \emph{fragments}) as well as first-order induction
and delocalisation by applying self-consistent procedure for 
optimisation DFI Hamiltonian matrix. The next very important
benefit is that it allows to study the interactions between 
\emph{more than two} molecules. Therefore it enables to take into 
account many-body effects. Due to the fact that previous methods 
are designed to calculate only coulomb iteraction energy, the 
DFI interaction energy is called \emph{pseudo-coulomb}.


% --------------------------------
\section{Auxiliary method overview\label{sec:aux}}
\subsection{Cumulative Atomic Multipole Moments}
Cumulative Atomic Multipole Moments (CAMMs) are  
cartesian atomic multipole moments distributed and centered 
at the atomic positions and can be expressed recursively 
by a change of coordinate system origin from the molecular 
center to the respective atomic centers:
\begin{equation}\label{e:CAMMs}
M_{klm,i}^{\rm CAMM} = M_{klm,i} - {\sum_{k'\ge 0}^{k\ne k'} 
                                    \sum_{l'\ge 0}^{l\ne l'} 
                                    \sum_{m'\ge 0}^{m\ne m'}} {k \choose k'} 
                                                              {l \choose l'} 
                                                              {m \choose m'} \times x_i^{k-k'} y_i^{l-l'} z_i^{m-m'} M_{k'l'm',i}^{\rm CAMM}
\end{equation}
The transformed atomic multipole moments $M_{klm,i}$ all 
centered at molecular origin and distributed at $i$-th atom 
are described as:
\begin{equation}\label{e:MMMs}
M_{klm,i} = Z_ix_i^ky_i^lz_i^m - \sum_{\mu\in i} \sum_{\nu} P_{\mu\nu} \tbraket{\mu}{x^ky^lz^m}{\nu}
\end{equation}
After applying equations \ref{e:CAMMs} and \ref{e:MMMs} 
for evaluation of the first four cumulative atomic multipoles
one obtains the following working formulae:
\begin{eqnarray}
q^i       &=& Z_i - \sum_{\mu\in i} \sum_{\nu} P_{\mu\nu} S_{\mu\nu} \\
\mu_{a}^i &=& \sum_{\mu\in i} \sum_{\nu} P_{\mu\nu} \big[    a_iS_{\mu\nu} -    \tbraket{\mu}{a}{\nu} \big] \\
Q_{ab}^i  &=& \sum_{\mu\in i} \sum_{\nu} P_{\mu\nu} \big[-a_ib_iS_{\mu\nu} + b_i\tbraket{\mu}{a}{\nu} + a_i\tbraket{\mu}{b}{\nu} 
                                                                             - \tbraket{\mu}{ab}{\nu} \big] \\
\Omega_{abc}^i &=& \sum_{\mu\in i} \sum_{\nu} P_{\mu\nu} \big[a_ib_ic_iS_{\mu\nu} - b_ic_i\tbraket{\mu}{a}{\nu} 
                                                                                  - a_ic_i\tbraket{\mu}{b}{\nu} 
                                                                                  - a_ib_i\tbraket{\mu}{c}{\nu} \\\nonumber
                                                               &&\qquad\qquad\quad+ c_i\tbraket{\mu}{ab}{\nu} 
                                                                                  + b_i\tbraket{\mu}{ac}{\nu}
                                                                                  + a_i\tbraket{\mu}{bc}{\nu}
                                                                             - \tbraket{\mu}{abc}{\nu} \big] 
\end{eqnarray} 
\subsubsection*{Traceless form of CAMMs}
Multpole $n$-rank tensor operators in 'ordinary' 
cartesian form are defined as follows:
\begin{equation}
{\mathbf m}^{(n)} = \sum_i e_i \underbrace{{\mathbf r}_i \otimes \cdots \otimes {\mathbf r}_i}_{n}
\end{equation}
For instance, dipole, quadrupole and octupole tensor 
operator elements are:
\begin{eqnarray}\label{e:quadord}
\mu_a         &=& \sum_i e_i a_i \\
 Q_{ab}       &=& \sum_i e_i a_i b_i \\
\Omega_{abc}  &=& \sum_i e_i a_i b_i c_i \quad  \textrm{where} \quad a,b,c=x,y \; \textrm{or} \; z
\end{eqnarray}
In many applications such a form of definition in the case
of quadrupole and higher multipole moments leads
to complicated formulas wich are difficult to operate
in practice. This problem can be solved by using so called
\emph{traceless form} of these operators which make them 
convenient in use. For quadrupole 
moment operator traceless forms is defined as follows:
\begin{equation}\label{e:quadtraceless}
Q_{ab}' = \frac{1}{2}\sum_i e_i \big( 3a_ib_i-r_i^2\delta_{ab}\big)
\end{equation}
In the case of octupole moment the definition is a bit more 
complex:
\begin{equation}
\Omega_{abc}' = \frac{1}{2}\sum_i e_i \big( 5a_ib_ic_i-r_i^2(a_i\delta_{bc}+
                                                                     b_i\delta_{ac}+
                                                                     c_i\delta_{ab})\big)
\end{equation}
In the above equations $r_i^2$ equals:
\begin{equation}\label{e:r2}
r_i^2 = \sum_{a=\{x,y,z\}} {a_i}^2  
\end{equation}
It is very easy to switch between 'ordinary' tensor form to 
traceless one and \emph{vice versa}. The transition to traceless
quadrupole moment from ordinary cartesian quadrupole moment 
can be done like in the example below. Rewriting the equation
\ref{e:quadtraceless} and substituting the definition from 
\ref{e:quadord} and \ref{e:r2} one quickly obtains:
\begin{eqnarray}
Q_{ab}' &=& \frac{3}{2}\sum_i e_ia_ib_i - \frac{1}{2}\sum_i\sum_{\alpha}e_i\alpha_i^2\delta_{ab} 
= \frac{3}{2}Q_{ab} -\delta_{ab}\sum_{\alpha}\sum_ie_i\alpha_i^2 =\\ \nonumber 
&=& \frac{3}{2}Q_{ab} - \delta_{ab}\;\mathrm{Tr}\;Q_{ab} 
\end{eqnarray}
Similar calculations for traceless octupole moment give
following expression:
\begin{eqnarray}
\Omega_{abc}' &=& \frac{5}{2}\sum_i e_ia_ib_ic_i - \frac{1}{2}\sum_i\sum_{\alpha}e_i\alpha_i^2 \big(
                                   a_i\delta_{bc} + b_i\delta_{ac} + c_i\delta_{ab}  \big) =\\ \nonumber
&=& \frac{5}{2}\Omega_{abc} - \frac{1}{2} \big( \Omega_{\alpha\alpha a}\delta_{bc} +
                                                \Omega_{\alpha\alpha b}\delta_{ac} + 
                                                \Omega_{\alpha\alpha c}\delta_{ab}  \big)
\end{eqnarray}
where $\Omega_{\alpha\alpha c}$ denotes trace of $\mathbf{\Omega}$ 
tensor with respect to the first two axes written in Einstein
summation notation for simplicity. The monopole, dipole, 
quadrupole and octupole moments from equations 3, 4, 13 and 14 
are used in standard {\sc COULOMB.py} routines for the calculations 
of electrostatic potential $V(\mathbf{r})$ as well as multipole part 
of electrostatic interaction energy $\Delta E^{\textrm{MTP}}_{el}$ 
between two multipole distributions. For more details see the section
\ref{sec:eint} about methods of calculations of electrostatic potential 
and interaction energy.

\subsection{Fitting charges from electrostatic potential}
The ESP method is used for calculation of coulomb interaction
between charge distributions or in assigning the partial charges 
on atoms in force field evaluation for molecular dynamics purposes. 
The key problem in ESP is to perform the best possible fitting of 
the charges to obtain the reference potential (in {\sc COULOMB.py} 
either from CAMMs or wave function). 

The function $\mathscr{Z}=\big| {\mathbf V}_{\rm ref} - {\mathbf A} {\mathbf q}^{\rm T}\big|$ 
has to be minimized, where ${\mathbf V}_{\rm ref}$ is reference 
potential, ${\mathbf q}$ is a vector of charges treated as variational 
parameters and ${\mathbf A}$ is the distance matrix between these charges, 
$A_{ij}=1/r_{ij}$. The convenient way to find ${\mathbf q}$ leads through 
least-square procedure. The squares of the difference of calculated 
potential has to be minimized with a constraint of constant total charge:
\begin{equation}\label{e:Zfunc}
Z(\{q_j\}) = \sum_i^m \Big(V_i-\sum_j^N \frac{q_j}{r_{ij}} \Big)^2 + \lambda\Big(\sum_j^N q_j-q_{tot}\Big)
\end{equation}
There is $m$ points in which electrostatic potential has to be
estimated\footnote{in {\sc COULOMB.py} either from CAMMs or 
directly from wave function. See section \ref{sec:eint}.} and $N$ 
charges $q_j$ to be fit (usually on atoms). The constraint
of the least-square fitting is here the total charge, $q_{tot}$
with the associated undetermined Lagrange multiplier, $\lambda$.

Minimisation of the function $Z(\{q_j\})$ is equivalent to take the
first derivatives of $Z$ with respect to fitted charges $q_k$ 
and to make it be equal to zero:
\begin{equation}
\Bigg(\frac{\pd Z(\{q_j\})}{\pd q_k} \Bigg)_{q_{(i\ne k)}}= 0
\end{equation}
Thus we have:
\begin{eqnarray}
\frac{\pd}{\pd q_k} 
 \Bigg[
     \sum_i^m \Big(
                  V_i-\sum_j^N \frac{q_j}{r_{ij}} \Big)^2 + 
                  \lambda\Big(\sum_j^N q_j-q_{tot}
              \Big) 
 \Bigg] = 
2\sum_i^m \Big(
                  V_i-\sum_j^N \frac{q_j}{r_{ij}} \Big) \Big( -\frac{1}{r_{ik}}\Big) + 
                  \lambda = \\\nonumber
\sum_j^N \sum_i^m \frac{q_j}{r_{ij}r_{ik}} - \sum_i^m \frac{V_i}{r_{ik}} + \lambda = 0 
\qquad\qquad\qquad\qquad\qquad\qquad\qquad\qquad\qquad\qquad\quad
\end{eqnarray}
This equation can be written in a matrix form as follows:
\begin{eqnarray}
\Bigl(\begin{matrix}
\bf A & 1\\ 
1     & 0  
\end{matrix}\Bigr) \Bigl(\begin{matrix}
\bf q \\ \lambda 
\end{matrix}\Bigr) = \Bigl(\begin{matrix}
\bf B \\ 0 
\end{matrix}\Bigr) \; ,
\end{eqnarray}
where the matrix elements are as follows:
\begin{eqnarray}
A_{jk} = \sum_i^m \frac{1}{r_{ij}r_{ik}} \\
B_k    = \sum_i^m \frac{V_i}{r_{ik}}     
\end{eqnarray}
Inverting the matrix thad multiplies the vector with charges
and $\lambda$ we get the fitted charges as a result.

\subsection{Density Fragment Interaction Method}

The DFI method is based on Hartree-Fock-Roothaan-Hall 
equasion, when Fock matrix is diagonalized iteratively
to obtain self-consistent density matrix and other useful
variables. The DFI scheme extends Fock matrix by taking 
into account the interactions between some pieces of an 
entire system called \emph{fragments} as well as bath-system
interactions. The entire Fock matrix comprises of fragment
block Fock matrices. Thus, the modified Fock matrix for 
a given fragment $A$ can be written as follows:
\begin{equation}
{\bf F}_A^{\rm DFI} = {\bf F}_A + \sum_{B\neq A} {\bf V}_{AB} + {\bf V}_{A-{\rm bath}}
\end{equation}
where ${\bf F}_A$ stands for gas-phase 'ordinary' Fock matrix,
${\bf V}_{AB}$ is the $A$-$B$-interfragment interaction and
${\bf V}_{A-{\rm bath}}$ denotes $A$-bath coupling. The explicit 
formulae for the respective interactions are:
\begin{eqnarray}
{\bf V}_{AB} &=& {\BM \mu}_{AB} + {\BM \nu}_{AB}\quad \rm{where} \\
{\BM \mu}_{AB} &=& -\sum_{b\in B} {\bf V}_A^b         \\
{\BM \nu}_{AB} &=& {\bf P_B} ({\bf J}(BA) - \frac{1}{2}{\bf K}(BA)) \\
J(BA)_{\lambda\sigma\in B}^{\mu\nu\in A} &=& (\mu\nu|\lambda\sigma) \\
K(BA)_{\lambda\sigma\in B}^{\mu\nu\in A} &=& (\mu\sigma|\lambda\nu)
\end{eqnarray}

DFI scheme adopts self consistent procedure to consider equally each 
of taken fragments. It is performed as follows:
\begin{enumerate}
\item ${\bf F}_A^{\rm DFI}$ for fragment $A$ is calculated 
      from separate gas-phase density matrices ${\bf P}_i$
      of fragments $i$. ${\bf P}_A$ density matrix is temporally
      set to zero in order to avoid self-coulombic interactions
      of this fragment.
\item SCF procedure is applied on ${\bf F}_A^{\rm DFI}$ and new,
      DFI density matrix for fragment A is obtained, ${\bf P}_A^{\rm DFI}$
\item We switch to the next fragment, $B$. We construct for it 
      ${\bf F}_B^{\rm DFI}$ from ${\bf P}_A^{\rm DFI}$,  ${\bf P}_B$ taken
      as zero and other ${\bf P}_i$. 
\item SCF procedure is applied on ${\bf F}_B^{\rm DFI}$ and now new,
      DFI density matrix for fragment B is obtained, ${\bf P}_B^{\rm DFI}$
\item We iterate like this until each of the fragments has been considered.
\end{enumerate}
After this procedure the respective DFI density matrices 
${\bf P}_i^{\rm DFI}$ as well as interfragment interaction 
potentials ${\bf V}_{ij}$ are obtained.  


\end{document}
=============================================================================================================================================
                                                               BROOLLOUIN
=============================================================================================================================================
